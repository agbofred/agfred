\documentclass[pdf, aspectratio=169, 12pt]{beamer}
\usepackage[]{hyperref, graphicx, siunitx, lmodern, tikz, booktabs, physics}
\usepackage[mode=buildnew]{standalone}
\usepackage{pdfpc-commands}
\usepackage{pgfplots}
\pgfplotsset{compat=1.16}

\usetheme[sols]{Python}

\graphicspath{ {Images/} }

\sisetup{per-mode=symbol}
\usetikzlibrary{calc, patterns, decorations.markings, decorations.pathmorphing, shapes}

%Preamble
\title{The 3 R's and a touch of imports}
\author{Jed Rembold}
\date{February 28, 2020}

\begin{document}

\begin{frame}{Announcements}
	\begin{itemize}
		\item Homework
			\begin{itemize}
				\item Homework 5 due tomorrow night
				\item I'm still working through the HW4 scoring
					\begin{itemize}
						\item Make sure you are filling out the commented metadata at the top of files please
					\end{itemize}
			\end{itemize}
		\item I do still plan on getting the grade reports out over the weekend. Sorry, its been a bit of a crazy week\ldots
		\item Polling: \url{rembold-class.ddns.net}
	\end{itemize}
\end{frame}

\begin{frame}[fragile]{Review Question}
	\begin{columns}
		\column{0.5\textwidth}
		Given the function and function call below, what would be printed?
		\column{0.5\textwidth}
		\begin{poll}
		\item \pyi{"My Name Is ABE"}
		\item \pyi{"MyNameIsABE"}
		\item \pyi{"My Name Is Abe"}
		\item \pyi{"mYnamEiSabE"}
		\end{poll}
	\end{columns}
	\exsol{\pyi{"MyNameIsABE"}}
	\vspace{-2mm}
	\begin{pythoncode}
		def rec(s):
			if len(s) <= 1:
				return s.upper()
			elif s[-2] == ' ':
				return rec(s[:-2]) + s[-1].upper()
			else:
				return rec(s[:-1]) + s[-1]

		print(rec("My name is ABE"))
	\end{pythoncode}
\end{frame}

\begin{frame}{Graphical Recursion}
	\begin{example}
		It was mentioned that recursion can show up in graphical applications as well. Let's try to reproduce the Sierpinski triangle in arcade!
	\end{example}
\end{frame}

\begin{frame}{Writing Example}
	\begin{example}
		Suppose I wanted to try my hand at some ASCII art and fill a text file with an oscillating vertical sine wave comprised of \pyi{o}'s. How could I generate and write it to file? How could I break the problem up into smaller bits? How could I make my solution more general and flexible?
	\end{example}
\end{frame}

\begin{frame}{Reading Example}
	\begin{example}
		I want to make a name generator which combines the first and last portions of the first names of random students in this class. I have all the first names in a text file called \texttt{Class.txt}. How could I do it? How can I break the problem up into smaller, more manageable chunks?
	\end{example}
\end{frame}

\begin{frame}{Appending (Writing More) Example}
	\begin{example}
		Often, you want to add some text to the end of an existing document. A case I had come up this summer was recording the temperature and humidity from a sensor. We don't have the sensor here, but let's construct how that would work with user input for the temperature and humidity.
	\end{example}
\end{frame}

%\begin{frame}{Going Global}
	%\begin{itemize}
		%\item<+-> Thus far, functions are localized
			%\begin{itemize}
				%\item Some inputs in the form of arguments
				%\item Some output in the from of what is returned
				%\item This is one of the major benefits of functions!
			%\end{itemize}
		%\item<+-> Sometimes though, we need some more flexibility
			%\begin{itemize}
				%\item Take our Fibonacci function. We would currently have no way to keep track of the number of times the function recursed
			%\end{itemize}
		%\item<+-> In these situations, one can use what are called \alert{global variables}
	%\end{itemize}
%\end{frame}

%\begin{frame}[fragile]{Global Details}
	%\vspace{5mm}
	%\begin{itemize}
		%\item A global variable is defined at the outermost scope of the program
			%\begin{itemize}
				%\item This can make it accessible to all functions in the program
			%\end{itemize}
		%\item To change a global variable, a function must declare explicitly within its code block that it wants to treat that variable name as a global variable
		%\item Syntax is: 
			%\begin{pythoncode}
				%def func(x):
					%global y
					%y += x

				%y = 2
				%func(5)
				%print(y)
			%\end{pythoncode}
	%\end{itemize}
%\end{frame}

%\begin{frame}[fragile]{HW4 Example}
	%\vspace{5mm}
	%\begin{example}
	%On HW4 there was a scoping issue with the below code. How could we fix it with a global variable?
	%\begin{pythoncode}[backgroundcolor=, numbers=left]
		%def odd_sum(maxnum=10):
			%def add_odd(x,total):
				%if x % 2 == 1:
					%total += x

			%total = 0
			%for n in range(maxnum+1):
				%add_odd(n,total)
			%return total
	%\end{pythoncode}
	%\end{example}
%\end{frame}

%\begin{frame}{With Great Power\ldots}
	%\begin{itemize}
		%\item The whole point of functions is to improve readability by keeping everything in a function local
		%\item Global variables destroy that localization
		%\item \textcolor{Red}{Wanton use of globals can wreck havoc on the clarity and readability of your code!}
			%\begin{itemize}
				%\item Use with caution and only where they are really needed
				%\item Can you achieve something without them? Then it is almost always better to do without
			%\end{itemize}
			
	%\end{itemize}
	
%\end{frame}

%\begin{frame}{\textcolor{Blue}{Import}ant Ideas}
	%\begin{itemize}
		%\item At some point, programs will reach a complexity where it would be nice to separate them over multiple files
			%\begin{itemize}
				%\item This helps with teamwork as well, since different people can easily work on different files then
			%\end{itemize}
		%\item Need some way to bring all those files together in a master program when the program is run
		%\item Python does this through the use of \pyi{import}
	%\end{itemize}
%\end{frame}

%\begin{frame}{Modules}
	%\begin{itemize}
		%\item A \alert{module} is a \texttt{.py} file which is imported into another program
		%\item Can consist of both executable statements as well as function definitions
			%\begin{itemize}
				%\item Usually mostly function definitions
				%\item Statements usually just for initialization
			%\end{itemize}
		%\item Can think of as a library, from which you can checkout a particular function that you want to use
		%\item \pyi{import}ing the module gives you access to all or some subset of those functions
	%\end{itemize}
%\end{frame}

%\begin{frame}{Usage example}
	%\begin{example}
		%Importing and using the module \texttt{circles.py}
	%\end{example}
%\end{frame}

%\begin{frame}{Namespaces}
	%\begin{itemize}
		%\item If just imported, functions and constants exist only in that modules namespace
			%\begin{itemize}
				%\item Need to refer to them with module name first, ie. \pyi{circles.area()}
			%\end{itemize}
		%\item This prevents conflicts with any other functions you might have also called \pyi{area()}!
	%\end{itemize}
%\end{frame}
\end{document}

