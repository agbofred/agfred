\documentclass[pdf, aspectratio=169, 12pt]{beamer}
\usepackage[]{hyperref, graphicx, siunitx, lmodern, tikz, booktabs, physics}
\usepackage[mode=buildnew]{standalone}
\usepackage{pdfpc-commands}

\usetheme[sols]{Python}

\graphicspath{ {Images/} }

\sisetup{per-mode=symbol}
\usetikzlibrary{calc, patterns, decorations.markings, decorations.pathmorphing, shapes}

%Preamble
\title{Getting Loopy}
\author{Jed Rembold}
\date{January 31, 2020}

\begin{document}

\begin{frame}{Announcements}
	\begin{itemize}
		\item Homework 1 is due tonight!
			\begin{itemize}
				\item Make sure you upload a pdf of your written work called HW1.pdf
				\item Make sure to change the assignment status to DONE in the assignment README.
				\item All changes should be committed to the same master branch
				\item If you have issues, let me know earlier rather than later!
				\item I'm aiming to grade these on Sunday, and I'll let you know when I'm done so you can check for any comments
			\end{itemize}
		\item Homework 2 will be posted later today
		\item Add/Drop deadline next Monday
		\item Polling: \url{rembold-class.ddns.net}
	\end{itemize}
\end{frame}

\begin{frame}{Some Musings\ldots}
		\emph{My experience has been that people good at following directions tend to be better at giving directions. Programming is all about giving directions. If you make a concerted effort to focus and improve your ability to follow directions, I think you'll be surprised at how it improves your coding!}
\end{frame}

\begin{frame}[fragile]{Review Question}
	Which of the below expressions could be assigned to \pyi{x} and would result in
	\begin{pythoncode}
		('hi all' * 3) != x
	\end{pythoncode}
	evaluating to \pyi{True}?
	\begin{poll}
	\item \pyi{x = 'hi all' + 'hi all' + 'hi all'}
	\item \pyi{x = 'hi allhi allhi all'}
	\item \pyi{x = str(3) + 'hi all'}
	\item \pyi{x = "hi" + 2 * ' allhi' + " all"}
	\end{poll}
	\exsol{\pyi{x = str(3) + 'hi all'}}

\end{frame}

\begin{frame}[fragile]{Getting Your Input}
	\begin{itemize}
		\item We already have \pyi{print()} for writing things to the terminal
		\item How can we give the program some sort of input short of editing the code?
			\begin{itemize}
				\item One method is the aptly named \pyi{input()} function
				\item Gives you a \alert{prompt} at the terminal to enter something
					\begin{pythoncode}
						guess = input('Guess any number between 1 and 10: ')
					\end{pythoncode}
				\item \emph{Always} returns a \alert{string}, so further conversion may be necessary
			\end{itemize}
	\end{itemize}
\end{frame}

\begin{frame}{Troubleshooting:}
	\begin{itemize}
		\item If your program reports an error when you run it:
			\begin{itemize}
				\item Look for the last line in the error message
				\item Frequently will give you some info on what bad thing happened and where in the code it happened
				\item Sometimes the actual mistake is not on that line, but immediately above it
			\end{itemize}
		\item If your program reports no error but a test is failing:
			\begin{itemize}
				\item Something is flawed in your set of instructions
				\item Walk yourself line by line through your code
				\item Frequently can help to \alert{hand-simulate} the code, writing down values and changing them as you move through your code.
					\begin{itemize}
						\item A great source to help you visualize this (and not by hand!) is \link{http://www.pythontutor.com/visualize.html\#mode=edit}{PythonTutor}!
					\end{itemize}
			\end{itemize}
	\end{itemize}
\end{frame}

\begin{frame}{If \pyi{if} fails}
	\begin{center}
		\includegraphics[width=0.5\textwidth]{lost_woods.png}
	\end{center}
	\begin{itemize}
		\item Moving to right (for instance) always resets to the same screen
		\item Moving any other direction reaches the exit
		\item How to code?
	\end{itemize}
\end{frame}

\begin{frame}[fragile]{Coding the Lost Woods}
	\vspace{5mm}
	\begin{pythoncode}
		if <move right>:
			<background = lost woods>
			if <move right>:
				<background = lost woods>
				if <move right>:
					<background = lost woods>
					...
				else:
					<background = exit>
			else:
				<background = exit>
		else:
			<background = exit>

	\end{pythoncode}
\end{frame}


\begin{frame}{Groundhog Day}
	\begin{center}
		\begin{tikzpicture}[
			block/.style={rounded corners, draw, very thick, fill=Orange, minimum width=2cm, minimum height=1cm, font=\Large, text=BG},
			choice/.style={rounded corners, draw, very thick, fill=Blue, minimum width=2cm, minimum height=1cm, font=\Large, diamond, text=BG}
			]
			\draw[line width=3pt, Purple] (1.5,-3) rectangle (9.5,3);
			\node[Purple, above right] at (1.5,-3) {Loop};

			\node[block](c1) at (-.5,0) {Code};
			\node[choice](t) at (3.5,0) {Test};
			\node[block, fill=Teal](loop) at (7,0) {Loop Body};
			\node[block](c2) at (11.5,0) {Code};

			\draw[ultra thick, -latex] (c1.east) -- (t.west);
			\draw[ultra thick, -latex, Teal] (t.east) |- (loop.west);
			\draw[ultra thick, -latex, Red] (t.north) |- ($(c2.west)+(-.5,1.5)$) |- (c2.west);
			\draw[ultra thick] (loop.east) -| +(.5,-1.5) -| (2,0);
		\end{tikzpicture}
	\end{center}
\end{frame}

\begin{frame}[fragile]{Python Notation}
	Multiple types of loop in Python, but looking at a \alert{while} loop first:
	\begin{pythoncode}
		<code>

		while <test>:
			<loop body>

		<code>
	\end{pythoncode}
\end{frame}

\begin{frame}[fragile]{Loop Simulations}
	Let's determine what the below code does by working through it by hand:
	\begin{pythoncode}
		x = 1
		total = 0
		while x < 10:
			if x % 2 == 1:
				total = total + x ** 2
			x = x + 1
	\end{pythoncode}
	\pause
	\begin{itemize}
		\item Should have found it adds the squares of all the odd numbers less than 10
		\item Result should have been 165
	\end{itemize}
\end{frame}

\begin{frame}{While Loop Tips}
	\begin{itemize}
		\item Make sure the body of your \pyi{while} loop is indented and has a colon after the condition
		\item \pyi{while} loops test some condition, so you may well need to define variable for that condition before the start of the loop.
		\item \emph{Make sure your loop will actually terminate at some point!} It can be easy to make infinite loops!
	\end{itemize}
	\begin{alertblock}{Infinite Loop Advice!}
		If you find yourself in an infinite loop, pressing Ctl+C in the terminal window will generally interrupt and stop the program!
	\end{alertblock}
\end{frame}


%\begin{frame}[fragile]{Understanding Check!}
	%\vspace{4mm}
	%What will the output of the below code be upon completion?
	%\begin{pythoncode}
		%x = 7
		%y = 0
		%while x >= 0:
			%y = y + x
			%x = x - 2
		%print(y)
	%\end{pythoncode}
	%\begin{poll}
		%\item 15
		%\item \alert<2>{16}
		%\item 28
		%\item Code will not complete (infinite loop)
	%\end{poll}
%\end{frame}

%\begin{frame}[fragile]{Lost Woods Examples}
	%\begin{columns}
		%\column{0.5\textwidth}
	%\begin{pythoncode}[style=output,]
		%************************
		%************************
					%:(
		%************************
		%************************
		%Go (L)eft or (R)ight?
	%\end{pythoncode}
		
		%\column{0.5\textwidth}
	%\begin{pythoncode}[style=output,]
		%**********     **********
		%**********     **********
					 %:)
		%**********     **********
		%**********     **********
		%Go (N)orth, (S)outh, 
		%(E)ast, or (W)est?
	%\end{pythoncode}
		
	%\end{columns}
%\end{frame}




\end{document}

