\documentclass[pdf, aspectratio=169, 12pt]{beamer}
\usepackage[]{hyperref, graphicx, siunitx, lmodern, tikz, booktabs, physics}
\usepackage[mode=buildnew]{standalone}
\usepackage{pdfpc-commands}
\usepackage{pgfplots}
\pgfplotsset{compat=1.16}

\usetheme[sols]{Python}

\graphicspath{ {Images/} }

\sisetup{per-mode=symbol}
\usetikzlibrary{calc, patterns, decorations.markings, decorations.pathmorphing, shapes}

%Preamble
\title{Look it up}
\author{Jed Rembold}
\date{March 11, 2020}

\begin{document}

\begin{frame}{Announcements}
	\begin{itemize}
		\item Homework
			\begin{itemize}
				\item Homework 7 
					\begin{itemize}
						\item Due on Friday
						\item Make sure to at least be getting started on Problem 2
					\end{itemize}
			\end{itemize}
		\item Midterm 1 week from Friday
			\begin{itemize}
				\item I'm going to try to get some study materials made up this weekend
				\item I will also post last semester's test
				\item No homework due next week so study up!
			\end{itemize}
		\item Polling: \url{rembold-class.ddns.net}
	\end{itemize}
\end{frame}

\begin{frame}{Review Question}
	What is the 2nd element (index 1) of the below list?
	\begin{center}
		\pyi{[(i-1)*j for i,j in enumerate('cow')]}
	\end{center}
	\begin{poll}
	\item \pyi{''}
	\item \pyi{'w'}
	\item \pyi{'o'}
	\item This would give an error
	\end{poll}
	\exsol{\pyi{''}}
\end{frame}


%\begin{frame}[fragile]{List Comprehensions Better Explained}
	%\begin{itemize}
		%\item Combines \pyi{for} loops and \pyi{list}s
		%\item If you don't want both, a list comprehension is probably not what you want
	%\end{itemize}
	%\vspace{-5mm}
	%\begin{center}
		%\begin{tikzpicture}[
			%block/.style={very thick, rounded corners, draw, minimum width=6cm, minimum height=5mm},
			%]
			%\node[text width=6cm, anchor=south](fl) at (0,0) {%
					%\begin{pythoncode}
						%L = []
						%for i in range(10):
							%if i % 2 == 0:
								%L.append(i**2)
					%\end{pythoncode}};
				%\node (lc) at (0,-2) {\pyi{[i**2\ \ for i in range(10) if i \% 2 ==0]}};

				%\onslide<2>{
					%\node[block, Blue] (c1) at (0,1.65) {};
					%\node[block, Blue, minimum width=10em] (lc1) at (-.8,-2) {};
					%\draw[thick, -stealth, Blue] (c1.east) -| ++(1,-2) -| (lc1.north);
				%}
				%\onslide<3>{
					%\node[block, Green] (c1) at (0,1.15) {};
					%\node[block, Green, minimum width=6em] (lc1) at (2.7,-2) {};
					%\draw[thick, -stealth, Green] (c1.east) -| ++(1,-2) -| (lc1.north);
				%}
				%\onslide<4>{
					%\node[block, SynPurple] (c1) at (0,0.65) {};
					%\node[block, SynPurple, minimum width=3em] (lc1) at (-3.4,-2) {};
					%\draw[thick, -stealth, SynPurple] (c1.east) -| ++(1,-2) -| (lc1.north);
				%}
		%\end{tikzpicture}
	%\end{center}
%\end{frame}


%\begin{frame}{Functional Objects}
	%\begin{itemize}
		%\item Functions are \alert{first-class objects}!
			%\begin{itemize}
				%\item Can treat them just like any other type of object (\pyi{int}, \pyi{list}, etc)
			%\end{itemize}
		%\item Can appear in
			%\begin{itemize}
				%\item expressions
				%\item arguments to other functions
				%\item elements of lists
				%\item etc
			%\end{itemize}
	%\end{itemize}
%\end{frame}

%\begin{frame}[fragile]{Follow the \pyi{map}}
	%\vspace{5mm}
	%\begin{itemize}
		%\item The builtin \pyi{map} function is like a more general purpose \pyi{apply_func_2_list}
		%\item Simplest just takes a single argument function and a list:
			%\begin{pythoncode}
				%def g(x):
					%return 4*x + 2

				%xs = [1,2,3,4,5,6]
				%for result in map(g,xs):
					%print(result)
			%\end{pythoncode}
	%\end{itemize}
%\end{frame}

%\begin{frame}[fragile]{Complex Mappings}
	%\begin{itemize}
		%\item Can take multiple argument functions if given multiple lists
			%\begin{pythoncode}
				%def h(x,y):
					%return 4*x + 2*y

				%xs = [1,2,3,4,5,6]
				%ys = xs[::-1]
				%hs = [res for res in map(h,xs,ys)]
				%print(hs)
			%\end{pythoncode}
	%\end{itemize}
%\end{frame}

\begin{frame}[fragile]{$\lambda$ functions}
	\begin{itemize}
		\item Often need a quick, one use function
		\item Annoying to have to define an entire function for that
		\item Can write \alert{anonymous functions} using \pyi{lambda} expressions
			\begin{itemize}
				\item Anatomy: \pyi{lambda <variables>: <expression>}
				\item Example:
					\begin{pythoncode}
						xs = [1,2,3,4,5]
						ys = [r for r in map(lambda x: x**2, xs)]
					\end{pythoncode}
			\end{itemize}
	\end{itemize}
\end{frame}

\begin{frame}{Understanding Check}
	One of the below expressions returns something different from the others. Which is the odd one out?
	\begin{poll}
	\item \pyi{[x+3 \ for x in range(10)]}
	\item \pyi{[x for x in map(lambda y: y + 3, range(10))]}
	\item \pyi{[lambda x: x+3 \ \ for x in range(10)]}
	\item \pyi{[y-3 \ for y in map(lambda x: x + 6, range(10))]}
	\end{poll}
	\exsol{\pyi{[lambda x: x+3 \ \ for x in range(10)]}}
\end{frame}

\begin{frame}[fragile]{Define it}
	\begin{itemize}
		\item A \alert{dictionary} is an \emph{unordered} set of objects
		\item Since unordered, we can't index them with a number
		\item Instead index them with \alert{keys}
		\item Any Python dictionary is a set of key/value pairings
		\item Delimited with \pyi{\{\}} to denote an unordered set
		\item Pairings use a \pyi{:} between key and value
			\begin{pythoncode}
				A = {'Jim':45, 'Bob':True, 34:'Betsy'}
			\end{pythoncode}
	\end{itemize}
\end{frame}

\begin{frame}[fragile]{A Key Concept}
	\begin{itemize}
		\item We still index dictionaries with square brackets
		\item Pass the desired \alert{key} inside
			\begin{pythoncode}
				print(A['Jim'])
				print(A[34])
			\end{pythoncode}
		\item Keys need to be unique!!
	\end{itemize}
\end{frame}

\begin{frame}{Dictionary Operations}
	\begin{itemize}
		\item Append:
			\begin{itemize}
				\item Just add a new entry
				\item \pyi{D['new key'] = 20}
			\end{itemize}
		\item Combine dictionaries:
			\begin{itemize}
				\item This extends a dictionary in place!
				\item \pyi{\{'A':1, 'B':2\}.update(\{'C':3, 'D':4\})}
			\end{itemize}
		\item Remove pairs:
			\begin{itemize}
				\item Removes a particular key/value pair
				\item \pyi{del D['key to remove']}
			\end{itemize}
		\item Loop over dictionary:
			\begin{itemize}
				\item Loops over the \alert{keys}
				\item \pyi{for key in \{'A':1, 'B':2, 'C':3\}:}
			\end{itemize}
	\end{itemize}
\end{frame}

\begin{frame}{Views}
	\begin{itemize}
		\item Can also access keys or values from a dictionary with special methods:
			\begin{itemize}
				\item Keys: \pyi{D.keys()}
				\item Values: \pyi{D.values()}
			\end{itemize}
		\item Both return a \alert{view object}, which is dynamic!
			\begin{itemize}
				\item Will change whenever the dictionary changes
			\end{itemize}
		\item Can loop over or test if things are in view objects
			\begin{itemize}
				\item Trying to change the size a dictionary while looping over it will result in an error
			\end{itemize}
	\end{itemize}
\end{frame}

\begin{frame}{Get it?}
	\begin{itemize}
		\item Commonly might want to access a key from a dictionary, but may be unsure if it exists
			\begin{itemize}
				\item Trying to access an invalid key will give an error
			\end{itemize}
		\item Use the \pyi{.get} dictionary method, which allows you to provide a fallback option if the desired key is not in the dictionary
			\begin{itemize}
				\item \pyi{D.get('desired key', 'fallback value')}
			\end{itemize}
	\end{itemize}
\end{frame}

%\begin{frame}[fragile]{Understanding Check}
	%What is the printed value?
		%{ \footnotesize
			%\begin{pythoncode}[tabsize = 2]
%A = [
		%{'name': 'Jill', 'weight':125, 'height':62},
		%{'name': 'Sam', 'weight':156, 'height':68},
		%{'name': 'Bobby', 'weight':173, 'height':75},
 %]
%A.append({'weight':204, 'height':70, 'name':'Jim'})
%B = A[1]
%B['weight'] = 167
%del A[0]['weight']
%print([d.get('weight',100) for d in A])
		%\end{pythoncode}
	%}
	%\begin{columns}
		%\column{0.4\textwidth}
		%\begin{poll}
		%\item \pyi{[100,167,173,204]}
		%\item \pyi{[100,156,173,204]}
		%\end{poll}
		
		%\column{0.4\textwidth}
		%\begin{poll}
			%\setcounter{enumi}{2}
		%\item \pyi{[125,167,173,204]}
		%\item \pyi{[100,156,173,70]}
		%\end{poll}
	%\end{columns}
%\end{frame}

%\begin{frame}{Sets}
	%\begin{itemize}
		%\item A \alert{set} is an unordered list of unique immutable objects
		%\item The set itself can be mutable (normally) or immutable (frozenset)
		%\item What works the same:
			%\begin{itemize}
				%\item The \pyi{in} function
				%\item Finding the length
				%\item Looping over elements
			%\end{itemize}
		%\item What breaks:
			%\begin{itemize}
				%\item No slicing or indexing!
			%\end{itemize}
		%\item Common uses:
			%\begin{itemize}
				%\item Removing duplicates from a sequence
				%\item Mathematical operations like intersection, union, difference
			%\end{itemize}
	%\end{itemize}
%\end{frame}

%\begin{frame}[fragile]{Non-Scalar Summary}
	%\begin{center}
		%\begin{tabular}{cccc}
			%\toprule
			%Type & Type of Elements & Examples of literals & Mutable \\
			%\midrule
			%\pyi{str} & characters & \pyi{'', 'a', 'abc'} & No \\
			%\pyi{tuple} & any type & \pyi{(), (2,), ('abc',3)} & No \\
			%\pyi{range} & integers & \pyi{range(5), range(2,10,2)} & No \\
			%\pyi{list} & any type & \pyi{[], [3], [6,'abc']} & Yes \\
			%\pyi{dict} & any type & \pyi{\{\}, \{'a':1\}, \{1:'abc', 2:5\}} & Yup \\
			%\pyi{set} & any immutable & \pyi{\{\}, \{2\}, \{'abc',5\}} & Both \\
			%\bottomrule
		%\end{tabular}
	%\end{center}
%\end{frame}



















\end{document}

