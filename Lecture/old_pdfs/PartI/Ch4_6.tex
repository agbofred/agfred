\documentclass[pdf, aspectratio=169, 12pt]{beamer}
\usepackage[]{hyperref, graphicx, siunitx, lmodern, tikz, booktabs, physics}
\usepackage[mode=buildnew]{standalone}
\usepackage{pdfpc-commands}
\usepackage{pgfplots}
\pgfplotsset{compat=1.16}

\usetheme{Python}

\graphicspath{ {Images/} }

\sisetup{per-mode=symbol}
\usetikzlibrary{calc, patterns, decorations.markings, decorations.pathmorphing, shapes}

%Preamble
\title{import tuples!}
\author{Jed Rembold}
\date{March 2, 2020}

\begin{document}

\begin{frame}{Announcements}
	\begin{itemize}
		\item Homework
			\begin{itemize}
				\item Homework 6 posted!
				\item Should already be able to do Prob 1 and 2
				\item Prob 3 should be doable by Wednesday
			\end{itemize}
		\item \alert{Grade reports actually got posted!}
			\begin{itemize}
				\item Includes everything that I currently have gotten graded.
				\item Today is last day to choose credit/no credit
			\end{itemize}
		\item Polling: \url{rembold-class.ddns.net}
	\end{itemize}
\end{frame}

\begin{frame}[fragile]{Review Question}
	\begin{columns}[b]
		\column{0.5\textwidth}
		To the right is a text file that was written to. Which of the below sample codes could have resulted in the text file to the right?
		\column{0.5\textwidth}
		\begin{pythoncode}[style=]
			Number: 34
			Number: 39
			Number: 44
		\end{pythoncode}
	\end{columns}
	\begin{columns}
		\scriptsize
		\column{0.5\textwidth}
		\begin{pythoncode}
			f = open('File.txt', 'w')
			for i in range(34,45,5):
				f.write('Number:', i, '\n')
			f.close()
		\end{pythoncode}
		\tikz[remember picture,overlay]{\node[font=\Large, Teal] at (6.5,1.25) {A};}
		\begin{pythoncode}
			f = open('File.txt', 'w')
			for i in range(34,45,5):
				f.write('Number: ' + i )
			f.close()
		\end{pythoncode}
		\tikz[remember picture,overlay]{\node[font=\Large, Teal] at (6.5,1.25) {B};}
		\column{0.5\textwidth}
		\begin{pythoncode}
			f = open('File.txt', 'w')
			for i in range(34,45,5):
				f.write('Number: '+str(i)+'\n')
			f.close()
		\end{pythoncode}
		\tikz[remember picture,overlay]{\node[font=\Large, Teal] at (6.5,1.25) {C};}
		\begin{pythoncode}
			f = open('File.txt', 'w')
			for i in range(34,45,5):
				f.write('Number:' + i + '\n')
			f.close()
		\end{pythoncode}
		\tikz[remember picture,overlay]{\node[font=\Large, Teal] at (6.5,1.25) {D};}
	\end{columns}
\end{frame}


%\begin{frame}[fragile]{Global Details}
	%\vspace{5mm}
	%\begin{itemize}
		%\item A global variable is defined at the outermost scope of the program
			%\begin{itemize}
				%\item This can make it accessible to all functions in the program
			%\end{itemize}
		%\item To change a global variable, a function must declare explicitly within its code block that it wants to treat that variable name as a global variable
		%\item Syntax is: 
			%\begin{pythoncode}
				%def func(x):
					%global y
					%y += x

				%y = 2
				%func(5)
				%print(y)
			%\end{pythoncode}
	%\end{itemize}
%\end{frame}

%\begin{frame}[fragile]{HW4 Example}
	%\vspace{5mm}
	%\begin{example}
	%On HW4 there was a scoping issue with the below code. How could we fix it with a global variable?
	%\begin{pythoncode}[backgroundcolor=, numbers=left]
		%def odd_sum(maxnum=10):
			%def add_odd(x,total):
				%if x % 2 == 1:
					%total += x

			%total = 0
			%for n in range(maxnum+1):
				%add_odd(n,total)
			%return total
	%\end{pythoncode}
	%\end{example}
%\end{frame}

%\begin{frame}{With Great Power\ldots}
	%\begin{itemize}
		%\item The whole point of functions is to improve readability by keeping everything in a function local
		%\item Global variables destroy that localization
		%\item \textcolor{Red}{Wanton use of globals can wreck havoc on the clarity and readability of your code!}
			%\begin{itemize}
				%\item Use with caution and only where they are really needed
				%\item Can you achieve something without them? Then it is almost always better to do without
			%\end{itemize}
	%\end{itemize}
%\end{frame}

\begin{frame}{\textcolor{Blue}{Import}ant Ideas}
	\begin{itemize}
		\item At some point, programs will reach a complexity where it would be nice to separate them over multiple files
			\begin{itemize}
				\item This helps with teamwork as well, since different people can easily work on different files then
			\end{itemize}
		\item Need some way to bring all those files together in a master program when the program is run
		\item Python does this through the use of \pyi{import}
	\end{itemize}
\end{frame}

\begin{frame}{Modules}
	\begin{itemize}
		\item A \alert{module} is a \texttt{.py} file which is imported into another program
		\item Can consist of both executable statements as well as function definitions
			\begin{itemize}
				\item Usually mostly function definitions
				\item Statements usually just for initialization
			\end{itemize}
		\item Can think of as a library, from which you can checkout a particular function that you want to use
		\item \pyi{import}ing the module gives you access to all or some subset of those functions
	\end{itemize}
\end{frame}

\begin{frame}{Usage example}
	\begin{example}
		Importing and using the module \texttt{circles.py}
	\end{example}
\end{frame}

\begin{frame}{Namespaces}
	\begin{itemize}
		\item If just imported, functions and constants exist only in that modules namespace
			\begin{itemize}
				\item Need to refer to them with module name first, ie. \pyi{circles.area()}
			\end{itemize}
		\item This prevents conflicts with any other functions you might have also called \pyi{area()}!
	\end{itemize}
\end{frame}

\begin{frame}{I'm Lazy}
	\begin{itemize}
		\item You can import in such a way that you don't have to use the module name
			\begin{itemize}
				\item Import only the function you want
					\begin{itemize}
						\item \pyi{from circles import area}
						\item Could then call normal as just \pyi{area(3)}
					\end{itemize}
				\item Import everything into global namespace
					\begin{itemize}
						\item \pyi{from circles import *}
						\item Then \emph{all} functions can be called without the module in front
					\end{itemize}
			\end{itemize}
		\item Be \textcolor{Red}{very} careful doing this!! Especially if you are importing a bunch of modules.
			\begin{itemize}
				\item Easy to accidentally override mission critical functions!
				\item Can make debugging a nightmare
			\end{itemize}
		\item I recommend just sticking to typing out the module name
	\end{itemize}
\end{frame}

\begin{frame}{The name is main}
	\begin{itemize}
		\item Sometimes you might want to write a script that serves multiple purposes
			\begin{itemize}
				\item Can be imported to give access to the defined functions
				\item Can be run directly to give some output
			\end{itemize}
		\item In these situations you can use
			\begin{center}
				\pyi{if __name__ == '__main__':}
			\end{center}
			\begin{itemize}
				\item Code inside that \pyi{if} statement will be run \alert{only} if the program is run directly, \alert{not} if it is imported
			\end{itemize}
	\end{itemize}
\end{frame}

\begin{frame}{What's your vector?}
	\begin{itemize}
		\item So far we've looked mostly at scalar variable types
			\begin{itemize}
				\item \pyi{int}
				\item \pyi{float}
				\item \pyi{bool}
				\item \pyi{str} $\leftarrow$ the only non-scalar type!
			\end{itemize}
		\item Chapter 5 is all about non-scalar variable types
			\begin{itemize}
				\item \pyi{tuple}
				\item \pyi{list}
				\item \pyi{range}
				\item \pyi{dict}
				\item \pyi{set}
			\end{itemize}
	\end{itemize}
\end{frame}

\begin{frame}{Introducing Tuples}
	\begin{itemize}
		\item<+-> Recall basic properties of a string
			\begin{itemize}
				\item Comprised of ordered smaller elements (characters)
				\item Immutable (can not be changed in place)
				\item Delimited by quotes 
			\end{itemize}
		\item<+-> Pythonic tuples: generalized strings
			\begin{itemize}
				\item Comprised of ordered smaller elements (of \emph{any} type!)
					\begin{itemize}
						\item Element variable types don't even need to be consistent!
					\end{itemize}
					
				\item Still immutable
				\item Delimited by parentheses
			\end{itemize}
	\end{itemize}
\end{frame}

\begin{frame}{Assigning Tuples}
	\begin{itemize}
		\item<+-> Place any sequence of variable types between parentheses, separated by commas
			\begin{itemize}
				\item \pyi{t_one = (1, 2, 3, 4)}
				\item \pyi{t_two = ('a', 'b', 'c')}
				\item \pyi{t_three = (1, 'a', 2, 'fish', True)}
			\end{itemize}
		\item<+-> Empty tuple just empty parentheses
			\begin{itemize}
				\item \pyi{t_empty = ()}
			\end{itemize}
		\item<+-> Tuple of 1 is the tricky one
			\begin{itemize}
				\item We already use parentheses to group together order of operation terms
				\item Even if you have only a single element, you \alert{need the trailing comma} to make it a tuple
				\item \pyi{t_single = ('a',)}
			\end{itemize}
	\end{itemize}
\end{frame}


\end{document}

