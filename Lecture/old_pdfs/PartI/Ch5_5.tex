\documentclass[pdf, aspectratio=169, 12pt]{beamer}
\usepackage[]{hyperref, graphicx, siunitx, lmodern, tikz, booktabs, physics}
\usepackage[mode=buildnew]{standalone}
\usepackage{pdfpc-commands}
\usepackage{pgfplots}
\pgfplotsset{compat=1.16}

\usetheme{Python}

\graphicspath{ {Images/} }

\sisetup{per-mode=symbol}
\usetikzlibrary{calc, patterns, decorations.markings, decorations.pathmorphing, shapes}

%Preamble
\title{A Set of Sprites}
\author{Jed Rembold}
\date{March 13, 2020}

\begin{document}

\begin{frame}{Announcements}
	\begin{itemize}
		\item Homework
			\begin{itemize}
				\item Homework 7 due tonight!
				\item No homework due next week!
			\end{itemize}
		\item Midterm 1 still week from Friday
			\begin{itemize}
				\item I'm going to get some study materials up by tomorrow (my plate is a bit full suddenly here\ldots)
			\end{itemize}
		\item Polling: \url{rembold-class.ddns.net}
	\end{itemize}
\end{frame}

\begin{frame}{The Elephant {\tiny(hopefully not Covid)} in the Room}
	\begin{itemize}
		\item Lectures will be streamed via Youtube or possibly Zoom
			\begin{itemize}
				\item Same time
				\item Will have chat interaction
				\item Still polling questions
			\end{itemize}
		\item Labs will probably switch over entirely to GitHub
			\begin{itemize}
				\item I'll aim to have some sort of video or voice conference running during those times to help with any questions
			\end{itemize}
		\item Homework will resume after Spring Break
		\item I'm still working out details on the test. I need the weekend.
		\item \alert{Pay attention to Campuswire}
			\begin{itemize}
				\item It will be where I make announcements of any changes
				\item Still the easiest place to ask and get questions answered
			\end{itemize}
	\end{itemize}
\end{frame}

\begin{frame}[fragile]{Review Question}
	What is the printed value?
		{ \footnotesize
			\begin{pythoncode}[tabsize = 2]
A = [
		{'name': 'Jill',  'weight':125, 'height':62},
		{'name': 'Sam',   'weight':156, 'height':68},
		{'name': 'Bobby', 'weight':173, 'height':75},
 ]
A.append({'weight':204, 'height':70, 'name':'Jim'})
B = A[1]
B['weight'] = 167
del A[0]['weight']
print([d.get('weight',100) for d in A])
		\end{pythoncode}
	}
	\begin{columns}
		\column{0.4\textwidth}
		\begin{poll}
		\item \pyi{[100,167,173,204]}
		\item \pyi{[100,156,173,204]}
		\end{poll}
		
		\column{0.4\textwidth}
		\begin{poll}
			\setcounter{enumi}{2}
		\item \pyi{[125,167,173,204]}
		\item \pyi{[100,156,173,70]}
		\end{poll}
	\end{columns}
	\exsol{A}
\end{frame}

\begin{frame}{Sets}
	\begin{itemize}
		\item A \alert{set} is an unordered list of unique immutable objects
		\item The set itself can be mutable (normally) or immutable (frozenset)
		\item What works the same:
			\begin{itemize}
				\item The \pyi{in} function
				\item Finding the length
				\item Looping over elements
			\end{itemize}
		\item What breaks:
			\begin{itemize}
				\item No slicing or indexing!
			\end{itemize}
		\item Common uses:
			\begin{itemize}
				\item Removing duplicates from a sequence
				\item Mathematical operations like intersection, union, difference
			\end{itemize}
	\end{itemize}
\end{frame}

\begin{frame}[fragile]{Non-Scalar Summary}
	\begin{center}
		\begin{tabular}{cccc}
			\toprule
			Type & Type of Elements & Examples of literals & Mutable \\
			\midrule
			\pyi{str} & characters & \pyi{'', 'a', 'abc'} & No \\
			\pyi{tuple} & any type & \pyi{(), (2,), ('abc',3)} & No \\
			\pyi{range} & integers & \pyi{range(5), range(2,10,2)} & No \\
			\pyi{list} & any type & \pyi{[], [3], [6,'abc']} & Yes \\
			\pyi{dict} & any type & \pyi{\{\}, \{'a':1\}, \{1:'abc', 2:5\}} & Yup \\
			\pyi{set} & any immutable & \pyi{\{\}, \{2\}, \{'abc',5\}} & Both \\
			\bottomrule
		\end{tabular}
	\end{center}
\end{frame}

\begin{frame}[fragile]{Getting Spritely}
	\begin{itemize}
		\item \pyi{SpriteList}s are common ways to display collections of images in arcade
		\item We probably need to understand how a single \pyi{Sprite} works first though!
		\item Basic syntax:
			\begin{pythoncode}
				arcade.Sprite(<filename>, <scale_factor>)
			\end{pythoncode}
		\item Image needs to be in the same folder (or a subfolder) of your script
		\item Can also set the center x and y positions at the same time.
		\item Draw to the screen with \pyi{.draw()}
		\item Lots of other built in capabilities in the form of other methods
	\end{itemize}
\end{frame}

\begin{frame}{Lists of Sprites}
	\begin{itemize}
		\item Frequently more useful to deal with groups of sprites
		\item Arcade also has a lot of performance enhancements for working with groups of sprites over single sprites
		\item Called a \pyi{SpriteList}
			\begin{itemize}
				\item Has similar methods to normal lists
					\begin{itemize}
						\item \pyi{.append}, \pyi{.extend}, \pyi{.remove}, etc
					\end{itemize}
				\item Has methods to draw or update every sprite in the list
				\item Can also be used for collision detections
			\end{itemize}
	\end{itemize}
\end{frame}

\begin{frame}{Animating Sprite Lists}
	\begin{itemize}
		\item Can combine to easily animate groups of Sprites
		\item Sprites can have properties which indicate how they should move:
			\begin{itemize}
				\item \pyi{change_x}
				\item \pyi{change_y}
				\item \pyi{velocity}
			\end{itemize}
		\item Calling an \pyi{.update} method will apply all these incremental movements
		\item \pyi{SpriteList}s can be drawn and updated all at the same time!
	\end{itemize}
\end{frame}





















\end{document}

