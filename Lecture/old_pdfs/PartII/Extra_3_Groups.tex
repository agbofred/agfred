\documentclass[pdf, aspectratio=169, 12pt]{beamer}
\usepackage[]{hyperref, graphicx, siunitx, lmodern, tikz, booktabs, physics, multicol}
\usepackage[mode=buildnew]{standalone}
\usepackage{pdfpc-commands}
\usepackage{pgfplots}
\pgfplotsset{compat=1.16}

\usetheme{Python}

\graphicspath{ {Images/} }

\sisetup{per-mode=symbol}
\usetikzlibrary{calc, patterns, decorations.markings, decorations.pathmorphing, shapes, fit}

%Preamble
\title{Collaborative Coding}
\author{Jed Rembold}
\date{April 20, 2020}

\begin{document}

\begin{frame}{Announcements}
	\begin{itemize}
		\item Homework
			\begin{itemize}
				\item If you haven't turned in any hw, try to get it in soon!
				\item I'm grading midterm projects, then hw, then late hw
			\end{itemize}
		\item Group Projects
			\begin{itemize}
				\item Emails with group assignments should have gone out. Start thinking through and talking what you want to create as a group!
				\item I'll be getting shared Github repositories made for each group to collaborate with. Will let you know once I have the details.
				\item Let me know if you are looking for ideas or want to bounce some ideas off me!
			\end{itemize}
		\item No class on Wednesday! Support the seniors by attending their talks virtually!
		\item Polling: \nolinkurl{rembold-class.ddns.net}
	\end{itemize}
\end{frame}

\begin{frame}{Approach to Large Projects}
	\vspace{5mm}
	\begin{enumerate}
		\item<+-> Have a clear idea of the core of what you want to build:
			\begin{itemize}
				\footnotesize
				\item Eg. A sofware terminal application to detect and record video and still images of shooting stars
			\end{itemize}
		\item<+-> Have clear goals for what you think success in the project would look like:
			\begin{itemize}
				\footnotesize
				\item System should reliably detect shooting stars
				\item System should periodically save current images of the night sky
				\item System should save video clips of shooting star events
				\item System processing should be able to run at real time framerates
				\item Starting, stopping, and changing settings should be accessible through a terminal interface
			\end{itemize}
	\end{enumerate}
	\begin{alertblock}{Important}<+->
		\begin{itemize}
			\item The more specific you can be, the easier life will be later.
			\item Everyone should know and be in agreement about these goals.
			\item Optional goals beyond the core features are totally ok.
		\end{itemize}
	\end{alertblock}
\end{frame}

\begin{frame}{Break It Up!}
	\begin{itemize}
		\item Divide goals into self-contained chunks of code
		\item Map out graphically how those chunks of code are related
			\begin{itemize}
				\item What depends on what? Or what will use objects defined in other code blocks?
				\item Be as specific as you can!
				\item Can help to make sure there is at least one chunk per group member, though more is fine
			\end{itemize}
	\end{itemize}
\end{frame}

\begin{frame}{Example Breakdown}
	\begin{center}
		\begin{tikzpicture}[
			block/.style={draw, very thick, rounded corners, minimum size=2cm, text width=3cm, align=center},
			v path/.style ={to path={|- (\tikztotarget)}}
			]
			\node[block, Red](detect) at (0,0) {Detection Algorithm and Functions};
			\node[block, Orange](vidsave) at (4,0) {Saving Video Clips};
			\node[block, Blue](imgsave) at (8,0) {Saving Sky Images};
			\node[block, Teal](control) at (4,-3) {Control and Main Script};
			\node[block, Red](gui) at (8,-3) {Terminal GUI};

			\path[ultra thick, -stealth, v path, draw, SynPurple]
				(detect.south) |- ([yshift=5mm]control.north) -- (control.north)
				(vidsave.south) -- (control.north)
				(imgsave) |- ([yshift=5mm]control.north) -- (control.north);
			\path[ultra thick, -stealth, v path, draw, SynPurple]
				(control.east) -- (gui.west);
		\end{tikzpicture}
	\end{center}
\end{frame}

\begin{frame}{Code Block Details}
	\begin{itemize}
		\item For each code chunk:
			\begin{itemize}
				\item List out what inputs you think you'll need
				\item Look at the calling code block and your goals to see what you should be returning
				\item Decide if the needed functionality can best be served by a set of functions or a class
					\begin{itemize}
						\item If a class, what getters and setters do you need? What methods?
						\item If a function(s), do you really want just a collection of functions? Or would a class be a better fit?
						\item Either way, \emph{include docstrings and documentation!}
					\end{itemize}
					
				\item Each code chunk should be functional given the proper inputs without the other pieces!
			\end{itemize}
	\end{itemize}
\end{frame}

\begin{frame}{Testing}
	\begin{itemize}
		\item Since code chunks should function independently, you should be able (and should!) test your chunk.
		\item Should test to make sure:
			\begin{itemize}
				\item Any desired goals are being met
				\item You are returning the correct or necessary things
			\end{itemize}
		\item Can write a few testing functions to help yourself if desired.
		\item Above all, just make sure you \textcolor{Red}{TEST IT!} Mistakes that come from your code are \emph{your} fault!
	\end{itemize}
\end{frame}

\begin{frame}{N'Sync}
	\begin{itemize}
		\item Need a method to reliably share your parts of the code with everyone
			\begin{itemize}
				\item Shared Github repositories are a good solution.
				\item Should check for updates from others each time before working on your portion
				\item Version control can save you if you manage to totally break something you once had working
			\end{itemize}
		\item Problems can be mitigated if everyone is working on a different file
		\item Use imports to bring everything together
		\item \textcolor{Red}{Don't wait until the last minute to upload your portion!}
			\begin{itemize}
				\item Errors or misunderstandings will crop up
				\item Give your group-mates time to react or adjust their parts to your code (or for you to adjust your parts to theirs)
			\end{itemize}
	\end{itemize}
\end{frame}








\end{document}

