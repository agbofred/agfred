\documentclass[pdf, aspectratio=169, 12pt]{beamer}
\usepackage[]{hyperref, graphicx, siunitx, lmodern, tikz, booktabs, physics, multicol}
\usepackage[mode=buildnew]{standalone}
\usepackage{pdfpc-commands}
\usepackage{pgfplots}
\pgfplotsset{compat=1.16}

\usetheme[sols]{Python}

\graphicspath{ {Images/} }

\sisetup{per-mode=symbol}
\usetikzlibrary{calc, patterns, decorations.markings, decorations.pathmorphing, shapes}

%Preamble
\title{Tis your Inheritance}
\author{Jed Rembold}
\date{April 6, 2020}

\begin{document}

\begin{frame}{Announcements}
	\begin{itemize}
		\item Homework
			\begin{itemize}
				\item I've posted HW9
				\item Both 8 and 9 are due on Friday
				\item I finished grading 6 yesterday and pushed results back to your Github
			\end{itemize}
		\item From what I've seen, I'm really excited to read over and score your midterm projects, but it will probably take me a week or so
		\item I'm planning to get grade reports given out for what I do have graded later today or tomorrow
		\item Polling: \nolinkurl{rembold-class.ddns.net}
	\end{itemize}
\end{frame}

\begin{frame}[fragile]{Review Question}
	\begin{columns}
		\column{0.5\textwidth}
		\small
		\begin{pythoncode}
			class Demo:
				def __init__(self):
					self.x = []
				def add(self,v):
					self.x.append(v)
				def get_x(self):
					return self.x

			A,B = Demo(), Demo()
			A.add(3)
			B.add(3)
			C = B.get_x()
			C.append(8)
			print(A.get_x() == B.get_x())
		\end{pythoncode}
		\column{0.5\textwidth}
		The code to the left defines a new class, creates some instances of that class and then manipulates them a bit. What is printed to the screen in the last line?
		\begin{poll}
		\item True
		\item False
		\item \pyi{None}, as \pyi{Demo} has no \pyi{__eq__} method
		\item This code would error out before completion
		\end{poll}
		\exsol{False}
		
	\end{columns}
\end{frame}

\begin{frame}{Reminders}
	\begin{itemize}
		\item We specify what a class's parent is immediately following the class's name
			\begin{itemize}
				\item Surround parent's name is parentheses
				\item \pyi{class Human(Animal):}
				\item The default parent is type \pyi{object}
			\end{itemize}
		\item All data attributes and methods defined for the parent are then defined for the child as well
			\begin{itemize}
				\item We are then free to:
					\begin{itemize}
						\item Add more methods
						\item Add more attributes
						\item Change inherited methods or attributes
					\end{itemize}
			\end{itemize}
	\end{itemize}
\end{frame}

\begin{frame}{A Swashbuckling Example}
	\begin{center}
		\begin{tikzpicture}[every node/.style={minimum size = 2cm, rounded corners, draw, ultra thick}]
			\node[Orange](human) at (0,0) {Human};
			\node[Red, right=1cm of human](pirate) {Pirate};
			\node[Blue, above right=0mm and 1cm of pirate](pegleg) {PegLeg};
			\node[Blue, below right=0mm and 1cm of pirate](patch) {Patch};

			\path[very thick, -stealth, Purple]
				(pegleg) edge (pirate)
				(patch) edge (pirate)
				(pirate) edge (human);
		\end{tikzpicture}
	\end{center}
\end{frame}

\begin{frame}{Adding or Changing Data Attributes}
	\begin{itemize}
		\item The \pyi{__init__} method is inherited from the parent class like all other methods!
		\item If we want other or more stuff to be initialized, we need to redefine \pyi{__init__}.
		\item Have some options in terms of how to do so:
			\begin{itemize}
				\item Redefine all the inherited attributes and then add our new attributes
				\item Call the parent \pyi{__init__} method and then add our own beneath
			\end{itemize}
	\end{itemize}
\end{frame}

\begin{frame}[fragile]{A Change is Coming}
	\begin{itemize}
		\item If redefining, ensure that all parent attributes are still present in the child!
			\begin{itemize}
				\item If you create a child object and plug it in somewhere in place of a parent object, it should still work!
			\end{itemize}
		\item To call the parent \pyi{__init__} method you need to access it through the \pyi{ClassName.__init__} structure
			\begin{itemize}
				\item Example:
					\begin{pythoncode}
						class Pirate(Human):
							def __init__(self, age): #redefining __init__
								Human.__init__(self, age)	 #calling parents __init__
					\end{pythoncode}
			\end{itemize}
	\end{itemize}
\end{frame}

\begin{frame}[fragile]{Understanding Check}
	\begin{columns}
		\column{0.5\textwidth}
		What expression would best fill in the gap in the code to the right?

		\small
		\begin{poll}
		\item \pyi{self.wage = wage}
		\item \pyi{TechJob.__init__(self, wage)}
		\item \pyi{TechJob.__init__(wage)}
		\item \pyi{Job.__init__(wage)}
		\end{poll}
		\exsol{\pyi{TechJob.__init__(self, wage)}}
		
		\column{0.55\textwidth}
		\footnotesize
		\begin{pythoncode}
			class Job(object):
				def __init__(self, wage):
					self.wage = wage

			class TechJob(Job):
				def __init__(self, wage):
					self.wage = wage
					self.code = True

			class SeniorDev(TechJob):
				def __init__(self, wage, exp):
					# What goes here?
					self.exp = exp
		\end{pythoncode}
	\end{columns}
\end{frame}

%\begin{frame}{Multiple Inheritance}
	%\begin{itemize}
		%\item A class is not limited to inheriting from only a single parent!
		%\item Can provide multiple parents!
			%\begin{itemize}
				%\item Ordering matters!
			%\end{itemize}
		%\item Example: a class that forms the intersection of two other types
	%\end{itemize}
%\end{frame}


\end{document}

