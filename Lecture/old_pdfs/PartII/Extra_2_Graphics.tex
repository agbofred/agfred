\documentclass[pdf, aspectratio=169, 12pt]{beamer}
\usepackage[]{hyperref, graphicx, siunitx, lmodern, tikz, booktabs, physics, multicol}
\usepackage[mode=buildnew]{standalone}
\usepackage{pdfpc-commands}
\usepackage{pgfplots}
\pgfplotsset{compat=1.16}

\usetheme{Python}

\graphicspath{ {Images/} }

\sisetup{per-mode=symbol}
\usetikzlibrary{calc, patterns, decorations.markings, decorations.pathmorphing, shapes, fit}

%Preamble
\title{All about Control}
\author{Jed Rembold}
\date{Nov 15, 2019}

\begin{document}

\begin{frame}{Announcements}
	\begin{itemize}
		\item Homework
			\begin{itemize}
				\item Homework 10 due tonight!
				\item Homework 9 was graded and results pushed back to your repositories
			\end{itemize}
		\item We'll talk projects the second half of class today
		\item Polling: \nolinkurl{rembold-class.ddns.net}
	\end{itemize}
\end{frame}

\begin{frame}[fragile]{Review Question}
	\vspace{4mm}
	The below code is run. Which image best depicts the state of the screen at the end of the program?
	\small
	\begin{pythoncode}
		w = g.GraphWin('Example', 400, 400)
		c = g.Circle(g.Point(100, 200), 20)
		c.draw(w)
		c.move(200, -100)
		c.move(0, 200)
		w.getMouse()
	\end{pythoncode}
	\begin{center}
		\vspace{-7mm}
		\begin{tikzpicture}
			\node(Ap) at (0,0) {\includegraphics[width=2.5cm]{A.png}};
			\node[below=1mm of Ap] {A};
			\node[right=of Ap] (Bp) {\includegraphics[width=2.5cm]{B.png}};
			\node[below=1mm of Bp] {B};
			\node[right=of Bp] (Cp) {\includegraphics[width=2.5cm]{C.png}};
			\node[below=1mm of Cp] {C};
			\node[right=of Cp] (Dp) {\includegraphics[width=2.5cm]{D.png}};
			\node[below=1mm of Dp] {D};
		\end{tikzpicture}
	\end{center}
\end{frame}


\begin{frame}{Understanding the Event Loop}
	\begin{itemize}
		\item When you create a \pyi{GraphWin} object, \pyi{graphics.py} starts running its own \alert{event loop}
		\item Runs alongside your code and controls the timing and actions that take place in the window
		\item Actions you make in your code get scheduled in the windows loop to get executed when the program has a chance
			\begin{itemize}
				\item If not much is going on, this is likely immediately
				\item If there is a LOT going on in your window, there might be a delay
			\end{itemize}
		\item \pyi{update()} generally brings the event loop and your code into agreement before moving on in each
		\item \pyi{.after()} schedules a task to take place at some future time in the event loop (when its schedule is free)
	\end{itemize}
\end{frame}

\begin{frame}{Blocking Events}
	\begin{itemize}
		\item \alert{Blocking events} will hold up \textcolor{Blue}{your code} (not the event loop) until that event happens
		\item The \pyi{.getMouse()} or \pyi{.getKey()} methods are blocking
			\begin{itemize}
				\item Python will not move onto the next line of your code until it either receives a mouse click or a key press!
				\item Essentially pauses your code until that happens (which may or may not be desired)
				\item Other things scheduled in the event loop may mean things are still happening in the window!
			\end{itemize}
	\end{itemize}
\end{frame}

\begin{frame}{Return to Sender}
	\begin{itemize}
		\item \pyi{.getMouse()} returns a \pyi{Point} object which contains the coordinates of where in the window a left click occurred
			\begin{itemize}
				\item Need to save the returned value as some variable to make use of it!
				\item Can access the individual elements of the \pyi{Point} object using \pyi{.getX()} or \pyi{.getY()}
				\item Only left click has an affect, and holding it down will not issue multiple clicks
			\end{itemize}
		\item \pyi{.getKey()} returns a string of the key that was pressed
			\begin{itemize}
				\item Again, need to save it as some variable to use it
				\item Don't know what string a certain key corresponds to? Just capture it and then print it out!
			\end{itemize}
	\end{itemize}
\end{frame}

\begin{frame}{Non-Blocking Events}
	\begin{itemize}
		\item \alert{Non-Blocking events} will \emph{not} stop your code when run
		\item The \pyi{.checkMouse()} and \pyi{.checkKey()} methods are non-blocking
			\begin{itemize}
				\item If no key or mouse is pressed when the command is run in your loop, \pyi{None} is returned instead
				\item Regardless, your loop continues on, no pausing
				\item If you are saving the value to be later used, make sure you handle \pyi{None} cases, since the majority of the time \pyi{None} is what will be returned
			\end{itemize}
	\end{itemize}
\end{frame}

\begin{frame}{Projects}
	\begin{itemize}
		\item Mostly groups of 3, which I have created and will be sending out
		\item Largely open to whatever you want or are interested in! I can help with ideas if you like.
		\item Will do a 10 min + questions presentation on the project on the last day of classes.
			\begin{itemize}
				\item Should include:
					\begin{itemize}
						\item What your project was
						\item How you approached it and divided up the problem
						\item Who worked on what parts
						\item How it came together and works
						\item Maybe a demonstration
					\end{itemize}
			\end{itemize}
		\item Code will be submitted to Github, but besides the presentation and code no other ``output'' necessary.
	\end{itemize}
\end{frame}

\begin{frame}{Timeline}
	\begin{itemize}
		\item Campuswire groups made for each grouping
		\item Monday's class will focus on planning a project and collaborative coding
		\item Wednesday we'll do some fun things with a program I wrote
		\item Before break, your group should have arrived at strong idea of what you want to do, how you'll go about doing that, and who is responsible for what.
		\item After break, probably both the Monday and the Wednesday class and lab time will be devoted to time you can work on the project and ask questions.
	\end{itemize}
\end{frame}


\end{document}

